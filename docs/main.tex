\documentclass[english, kiv, sem, he, iso690alph, pdf, viewonly]{fasthesis}
\title{I-node based Filesystem}
\author{Jan}{Hejdušek}
\supervisor{}
\assignment{sw2025-02.pdf}

\usepackage{csquotes}
\usepackage{pdfpages}

\nobastardtitle
\nocopyrightnotice

\newif\iffullbuild
\fullbuildtrue

\lstset{style=FASThesisLstStyle, numberblanklines=false, tabsize=5,
keywordstyle=\color{red}}

\begin{document}
\iffullbuild
\frontpages[notm]
\tableofcontents
\fi

\chapter{Introduction}
This document is a documentation to a seminar work for the course
\lstinline|KIV/ZOS|.

It describes the implementation of a simple virtual
filesystem based on the i-nodes. The filesystem simulates a disk
using a binary file and provides basic operations for file and
directory management. The goal of this project is to demonstrate
understanding of low-level filesystem structures.

The document covers the following topics:

\begin{itemize}
  \item Description of the internal disk layout and core data structures.
  \item Overview of the source code structure and the main roles of each module.
  \item User guide with instructions for compiling, running, and
    using the filesystem.
  \item Conclusion summarizing the achieved functionality,
    possible improvements and encountered difficulties.
\end{itemize}

\chapter{Implementation}

The filesystem is implemented in C and operates on a virtual disk
file. The disk structure is divided into sections:

\begin{itemize}
  \item \textbf{Superblock}: Contains global metadata about the
    filesystem, such as the disk size, block size, and pointers to
    other regions.
  \item \textbf{I-node Bitmap}: A bit array indicating which i-nodes
    are currently in use.
  \item \textbf{Data Bitmap}: A bit array indicating which data
    blocks are occupied.
  \item \textbf{I-nodes}: An array of i-node structures. Each i-node
    stores metadata for a file or directory (size, type, pointers to
    data blocks).
  \item \textbf{Data Clusters}: The actual storage area for file
    content and directory entries.
\end{itemize}

\section{Core Structures}

The implementation relies on several C structures. Some of those were
inspired by
a sample source codes provided by the course.

\subsection{Superblock}
The \texttt{superblock} structure stored at the beginning of the disk
and defines its layout. It includes necessary data for managing the
filesystem. It contains size of the disk, counts for i-nodes and
clusters, cluster size in bytes and
byte offsets pointing to the start of bitmaps, i-node table, and
data regions.

\subsection{I-node}
The \texttt{inode} structure stores following:
\begin{itemize}
  \item \texttt{node\_type}: To distinguish between files and directories.
  \item \texttt{file\_size}: The size of the file in bytes.
  \item \texttt{direct\_blocks}: A static array of cluster IDs.
  \item \texttt{indirect\_blocks}: Single and double indirect cluster IDs.
\end{itemize}

\section{Source Code Organization}

The source code is modularized to separate the user interface,
high-level command logic, and low-level filesystem operations.

\subsection{Main Entry Point (\texttt{main.c})}
This module serves as the application's entry point. It is
responsible for the initial setup of the environment. Specifically, it:
\begin{itemize}
  \item Validates command-line arguments to ensure a disk file path is provided.
  \item Opens the virtual disk file in binary read/write mode.
  \item Reads the superblock to verify the filesystem signature
    (\texttt{"HEJDULA"}).
  \item Prints the disk geometry (size, cluster count, etc.) upon startup.
  \item Initializes the global system state and hands over control to the REPL.
\end{itemize}

\subsection{Interactive Shell (\texttt{repl.c})}
The \texttt{repl.c} module implements the Read-Eval-Print Loop
(REPL), providing the command-line interface for the user. Its
primary responsibilities include:
\begin{itemize}
  \item \textbf{Input Parsing}: It reads user input, handles
    tokenization, and parses arguments.
  \item \textbf{Command Dispatch}: It matches the input token against
    a registry of available commands and invokes the corresponding function.
  \item \textbf{Error Handling}: It translates numeric error codes
    returned by commands into human-readable error messages.
  \item \textbf{Batch Processing}: It includes logic to read and
    execute commands line-by-line from an external script file via
    the \texttt{load} command.
\end{itemize}

\subsection{Core Filesystem Engine (\texttt{dulafs.c})}
This is the main module handling the low-level logic and internal functioning
of the filesystem. It abstracts binary file operations into logical
filesystem primitives. It is further divided into functional areas:

\subsubsection{Formatting the Virtual Disk}
Formatting sets up the main parts of the virtual disk file: the superblock,
bitmaps, inode table, and data clusters. It calculates where
each part goes and writes the superblock at the start of the file.
Formatting also initializes the root directory and its cluster.

It is possible to set a ratio of inodes to clusters and size of
idividual clusters.

\subsubsection{Bitmap Management}
Functions such as \texttt{set\_bit}, \texttt{clear\_bit}, and
\texttt{get\_empty\_index} directly manipulate the bits in the inode
and data bitmaps. This ensures efficient tracking of free and occupied space.

\subsubsection{Inode Operations}
This part of the code manages the life cycle and usage of i-nodes,
which are the main structures for storing information about files and
directories. The following operations work with i-nodes:

\begin{itemize}
  \item \textbf{Allocation}: Finds a free i-node using the i-node
    bitmap and marks it as used when a new file or directory is created.
  \item \textbf{Reading and Writing}: Loads i-node data from and
    saving data to the inode clusters.
  \item \textbf{Cluster Assignment}: Uses the
    \texttt{assign\_node\_clusters} function to allocate data blocks
    (clusters) for files, including handling direct, single indirect,
    and double indirect pointers for larger files.
  \item \textbf{Deletion}: Frees the i-node and its associated data
    blocks when a file or directory is deleted and updating the bitmaps.
  \item \textbf{Reference Counting}: Keeps track of how many
    directory entries point to an i-node, so it is only deleted when
    no references remain.
\end{itemize}

\subsubsection{Directory and Path Logic}
This part of the code manages how directories are organized and how
paths are resolved in the filesystem. It allows the system to work
with a tree-like structure of folders and files. It handles the following:

\begin{itemize}
  \item \textbf{Path Resolution}: Finds the i-node for a given path
    string (like \texttt{/home/user/file}) by walking through the
    directory tree step by step. The \texttt{path\_to\_inode}
    function is used for this.
  \item \textbf{Directory Entry Management}: Adds new files or
    directories to a folder, or removes them, by updating the list of
    directory entries. Functions like \texttt{add\_record\_to\_dir}
    and \texttt{delete\_item} handle these changes.
  \item \textbf{Parent and Self References}: Handles automatic
    creation of entries \texttt{.} (current directory) and
    \texttt{..} (parent directory).
  \item \textbf{Name Lookup}: Searches for a file or directory by
    name within a folder to check if it exists or to get its i-node.
  \item \textbf{Path Parsing}: Splits a path into its components to
    find parent directories or the final file/directory name.
\end{itemize}

\subsection{Command Implementation (\texttt{commands.c})}
This module contains the main logic for all user commands,
such as \texttt{ls}, \texttt{cp}, and \texttt{mkdir}.

All command functions have the same signature: they take the number
of arguments (\texttt{argc}) and an array of argument strings
(\texttt{argv}), and return an integer error code. The implemented
commands are stored in a structure that includes the command name, a
pointer to its function and the expected number of arguments (or
  \texttt{-1} to not check the number of args before calling the
command function).

\chapter{User guide}
\section{Compilation}

The program is designed to be portable across platforms with a
compliant C compiler. A Makefile is provided for easy compilation using GCC.

\section{Example Usage}

The application is a command-line tool that accepts a file to open as
an argument (the file must exist).

Below is an example session demonstrating opening an existing
formated virtual disk file to 20MB,
making a directory, copying in a file and printing its content.
Including also an invalid operation to showcase the error message.

\setuxprompt{hejdula@linux}{home}
\begin{console}{Example of running the program}
`\uxprompt`./dulafs.out vfsdisk.ula
Trying to open: vfsdisk.ula
Valid .ula filesystem detected!

=== Superblock Information ===
Signature: 'HEJDULA'
Disk size: 20971520 bytes
Cluster size: 4096 bytes
Cluster count: 5016
Inode count: 10485
Inode bitmap start address: 40
Cluster bitmap start address: 1351
Inode start address: 1978
Data start address: 421407
===============================

Welcome to dula REPL, available commands: format, cp, mv, rm,
mkdir, rmdir, ls, cat, cd, pwd, info, incp, outcp, load, statfs, ln, test
dulafs:/> mkdir dir
[ok] dulafs:/> ls
..           | inode:   0 | size:     48 bytes | refs: 0
.            | inode:   0 | size:     48 bytes | refs: 0
dir          | inode:   1 | size:     32 bytes | refs: 1
[ok] dulafs:/> incp hello dir/hi
[ok] dulafs:/> cat hi
Error: Path not found
[3] dulafs:/> cat dir/hi
hello
[ok] dulafs:/> exit
`\uxprompt`
\end{console}

\chapter{Conclusion}
\section{Functionality}
The filesystem works according to the assignment and supports all the
required basic file operations. It can create, delete, and list files
and directories, and manages free space. The system
supports hierarchical directories and can handle files of different
sizes using direct and indirect addressing. Overall, the main
features work as expected and the filesystem can be used for simple
file management tasks.

\section{Expectations}
While working on this project, I was surprised by how much overhead
is needed for a filesystem, even when it is designed to be as simple
as possible. There are many details to handle, such as keeping track
of free space, managing metadata, and making sure all the structures
are consistent. Even a minimal filesystem requires a lot of
supporting code and careful planning.

\section{Possible Improvements}
There are several areas where the project could be improved. I am not
fully satisfied with how errors are handled, sometimes the errors are
not fully propagated all the way and not every failing operation is
properly handled.
The way relative paths are resolved could also be made more robust
and simpler to use.
Another thing, the global variables/structures are badly structured
and it is not clear what exactly it does.

\section{Difficulties}
One of the main difficulties was that a lot of code had to be written
before I could even test if anything worked. Many parts of the system
depend on each other, so it was challenging to get everything working
together. I often had to refactor the code as I discovered better
ways to organize it or found bugs.

Handling all the edge cases and
figuring out where things could fail also took a lot of time and
effort.
\end{document}

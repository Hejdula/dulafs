\documentclass[english, kiv, sem, he, iso690alph, pdf, viewonly]{fasthesis}
\title{I-node based Filesystem}
\author{Jan}{Hejdušek}
\supervisor{}
\assignment{sw2025-02.pdf}

\usepackage{csquotes}
\usepackage{pdfpages}

\nobastardtitle
\nocopyrightnotice

\newif\iffullbuild
\fullbuildtrue

\lstset{style=FASThesisLstStyle, numberblanklines=false, tabsize=5,
keywordstyle=\color{red}}

\begin{document}
\iffullbuild
\frontpages[notm]
\tableofcontents
\fi

\chapter{Introduction}
\chapter{Analysis}
\chapter{Implementation}
\chapter{User guide}
\section{Compilation}

The program is designed to be
portable across platforms with a compliant C compiler. While the
application can be built using a standard toolchain, makefiles for
the GCC compiler are
provided for both GNU/Linux and Microsoft Windows environments.

\section{Execution}

The application supports two modes of operation: interactive mode and
batch processing.
Executing the program without arguments initiates the interactive
interpreter (REPL).
Alternatively, providing a file path as an argument processes the
contents of the specified file with Lisp source code.
A verbose execution mode, which prints results of outer expressions,
can be enabled by appending the \texttt{-v} flag when running with an
input file. Example of running the program in a Linux environment on a
Lisp source code with bubble sort\footnote{The Lisp source code for
  the bubble sort algorithm was taken from the assignment at the end of
this document} algorithm:

\setuxprompt{hejdula@linux}{home}
\begin{console}{Example of running the program}
`\uxprompt` clisp.exe path/to/bubble_sort.lisp -v
[1]> (SET (QUOTE ARR) (QUOTE (2 5 1 9 0 8 3 7 4 6)))
(2 5 1 9 0 8 3 7 4 6)
[2]> (SET (QUOTE SWAPPED) T)
T
[3]> (SET (QUOTE LEN) (LENGTH ARR))
10
[4]> (WHILE SWAPPED (SET (QUOTE I) 1) (SET (QUOTE SWAPPED) NIL)
  (WHILE (< I LEN) (SET (QUOTE NUM1) (NTH (- I 1) ARR)) (SET (QUOTE
    NUM2) (NTH I ARR)) (IF (> NUM1 NUM2) (WHILE T (SET (NTH (- I 1)
ARR) NUM2) (SET (NTH I ARR) NUM1) (SET (QUOTE SWAPPED) T) (BRK))) (INC I 1)))
NIL
(0 1 2 3 4 5 6 7 8 9)
[5]> (PRINT ARR)
(0 1 2 3 4 5 6 7 8 9)
`\uxprompt`
\end{console}

\chapter{Conclusion}
\section{Functionality}
\subsection{Test Runs}
\section{Expectations}
\section{Possible Improvements}
\section{Difficulties}

\end{document}
